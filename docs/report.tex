\documentclass[a4paper, titlepage, 12pt]{article}
\usepackage{graphicx}
\usepackage{color}
\usepackage{hyperref}
\usepackage{amsmath}
\usepackage{amsthm}
\usepackage{adjustbox}
\usepackage{float}
\usepackage{mdframed}
\usepackage{subfig}
\DeclareMathOperator{\argmax}{arg\,max}
\numberwithin{equation}{section}
% For box
\usepackage{collectbox}

\makeatletter
\newcommand{\mybox}{%
    \collectbox{%
        \setlength{\fboxsep}{1pt}%
        \fbox{\BOXCONTENT}%
    }%
}
\makeatother
%Endfor box



\begin{document}
    \title{ECE - 5534 Electronic Design Automation \\ ROBDDs}
    \author{Prathamesh Mandke \\ \href{pkmandke@vt.edu}{pkmandke@vt.edu} \\ PID: 906239574}
    \date{\today \\ \href{https://ece.vt.edu}{ECE} @ \href{https://vt.edu}{Virginia Tech}}
    \maketitle
    \newpage

    \section{Lab Overview}

        Reduced Ordered Binary Decision Diagrams (ROBDDs) are concise representations of boolean expressions using a directed acyclic graph.
        ROBDDs make use of the if-then-else (INF) form of boolean expressions to construct the graphical representation.
        An INF form of a boolean expression is one where any operations are only perfomed on variables with constants and INF statements.
        The practicality of using the INF form to construct ROBDDs for any boolean expression stems from the Shannon Expansion idea.
        Shannon Expansion involves breaking down an expression (with arbitrary number of variables) into an INF form by recursively setting and resetting the value of one variable at a time.
        From the perspective of implementation in digital logic, the Shannon Expansion involves cascading multiplexer circuits by considering all combinations of a boolean variable at a time.
        
        When a complicated boolean expression is recursively broken down into it's INF, a series of INF statements are obtained.
        Now, if any redundant INF statements are combined, the resulting set of INF statements can be expressed as a binary decision diagram.
        When the order of variable selection is same across all recursive builds of the INF forms, the resulting decision diagram has nodes in the same order starting from the root.
        Such a BDD is said to be an ordered BDD.
        Going one step further, multiple nodes with the same left and right nodes can be combined since they are obviously redundant.
        Such reduction leads to a unique representation of the boolean expression known as the ROBDD.
        However, it is interesting to note that this representation is not unique to the order of variable selection and is in fact highly sensitive to it.
        
        This work involves the implementation of a ROBDD for boolean operations AND, OR, NOT, IMPLICATION and EQUIVALENCE.
        The methods used to implement the ROBDD have been inspired from Henrik Anderson's document on Binary Decision Diagrams.

    \section{Tests}

        This section involves multiple tests and their results for verifying the functionality.
        Initially, I have considered simple unit tests for verifying the correctness of individual methods such as Build, Apply, etc. 
        Further, there are somewhat non-trivial and nuanced test cases for demonstrating the usefulness, simplicity and effectiveness of ROBDDs.

        \subsection{Unit Tests}

            This section includes unit tests for verifying functionality of individual methods of the ROBDD implementation.

            \begin{itemize}
                \item[1.] \tetxbf{Build (and Make)}
                                        
            \end{itemize}

\end{document}